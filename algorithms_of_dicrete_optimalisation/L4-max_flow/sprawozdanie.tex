\documentclass[a4paper,11pt]{article}
\usepackage[polish]{babel}
\usepackage[utf8]{inputenc}
\usepackage[T1]{fontenc}
\usepackage{graphicx}
\usepackage{float}
\usepackage{amsmath}
\usepackage{geometry}
\usepackage{hyperref}

\geometry{margin=2.5cm}

\title{Sprawozdanie: Algorytmy Optymalizacji Dyskretnej - Lista 4}
\author{Imię Nazwisko (Nr Indeksu)}
\date{\today}

\begin{document}

\maketitle

\section{Wstęp}
Celem niniejszego sprawozdania jest analiza i przedstawienie wyników implementacji algorytmów maksymalnego przepływu oraz skojarzeń w grafach dwudzielnych. Zadania obejmowały implementację algorytmów Edmonds-Karp oraz relabel-to-front (lub wariantu najkrótszej ścieżki powiększającej - SAP), modelowanie problemu jako Programowanie Liniowe (LP) oraz badanie skojarzeń w grafach losowych.

\section{Opis implementacji algorytmów (Zadania 1, 2, 4)}

\subsection{Struktura FlowNetwork}
Podstawą implementacji jest klasa \texttt{FlowNetwork}, reprezentująca sieć przepływową za pomocą list sąsiedztwa. Każda krawędź przechowuje informację o wierzchołku docelowym, aktualnej przepustowości rezydualnej, indeksie krawędzi powrotnej oraz oryginalnej przepustowości.

\subsection{Algorytm Edmondsa-Karpa}
Zaimplementowano klasyczny algorytm Edmondsa-Karpa, który jest realizacją metody Forda-Fulkersona wykorzystującą przeszukiwanie wszerz (BFS) do znajdowania najkrótszych ścieżek powiększających w grafie rezydualnym.
\begin{itemize}
    \item \textbf{Złożoność}: $O(V E^2)$.
    \item \textbf{Działanie}: W każdej iteracji BFS znajduje najkrótszą ścieżkę (w sensie liczby krawędzi) z źródła do ujścia. Przepływ jest zwiększany o "wąskie gardło" tej ścieżki.
\end{itemize}

\subsection{Algorytm Shortest Augmenting Path (SAP)}
Jako algorytm alternatywny użyto podejścia 'Shortest Augmenting Path' (implementacja wzorowana na metodach typu Goldberg-Tarjan/Relabel, ale z wykorzystaniem etykiet odległości do szybkiego znajdowania dopuszczalnych krawędzi).
\begin{itemize}
    \item \textbf{Złożoność}: Teoretycznie $O(V^2 E)$ dla ogólnej metody, w praktyce często znacznie szybszy od EK na gęstych grafach.
    \item \textbf{Optymalizacje}: Użycie "Gap Heuristic" oraz wskaźników \texttt{current\_arc} pozwala na unikanie ponownego przeglądania nasyconych krawędzi.
\end{itemize}

\subsection{Maksymalne skojarzenie w grafie dwudzielnym (Zadanie 2)}
Problem znalezienia maksymalnego skojarzenia w grafie dwudzielnym zredukowano do problemu maksymalnego przepływu:
1. Dodano super-źródło $S$ połączone krawędziami o pojemności 1 ze wszystkimi wierzchołkami zbioru $V_1$.
2. Dodano super-ujście $T$, do którego wchodzą krawędzie o pojemności 1 od wszystkich wierzchołków zbioru $V_2$.
3. Istniejące krawędzie między $V_1$ a $V_2$ otrzymały pojemność 1 (lub $\infty$).
4. Maksymalny przepływ w takiej sieci odpowiada liczności maksymalnego skojarzenia (z uwagi na całkowitoliczbowość przepływu).

\section{Opis modeli Programowania Liniowego (Zadanie 3)}

Dla problemu maksymalnego przepływu wygenerowano modele w formacie LP (rozwiązywalne przez GLPK).

\subsection{Zmienne decyzyjne}
Dla każdej krawędzi skierowanej $(u, v)$ w grafie zdefiniowano zmienną decyzyjną $f_{u,v}$ oznaczającą wielkość przepływu na tej krawędzi.

\subsection{Funkcja celu}
Celem jest maksymalizacja wypływu ze źródła $s$:
$$ \max \sum_{v \in N(s)} f_{s,v} - \sum_{u \in N^{-1}(s)} f_{u,s} $$
(W praktyce często wystarczy suma wypływów, jeśli do źródła nic nie wraca).

\subsection{Ograniczenia}
\begin{enumerate}
    \item \textbf{Ograniczenie przepustowości}: Dla każdej krawędzi $(u, v)$, przepływ nie może przekraczać przepustowości $c_{u,v}$:
    $$ 0 \le f_{u,v} \le c_{u,v} $$
    \item \textbf{Zachowanie przepływu}: Dla każdego węzła $v \in V \setminus \{s, t\}$, suma wpływów musi równać się sumie wypływów:
    $$ \sum_{u} f_{u,v} = \sum_{w} f_{v,w} $$
\end{enumerate}

Model generowany przez program zapisuje te równania w standardzie CPLEX LP, np.:
\begin{verbatim}
Maximize
 obj: f_0_1 + f_0_2
Subject To
 c1: f_0_1 - f_1_3 = 0
 ...
\end{verbatim}

\section{Wyniki eksperymentów}

\subsection{Zadanie 1: Przepływ w hipersześcianie}
Przeprowadzono testy dla wymiarów $k$ od 1 do 16. Porównano czas działania algorytmów Edmonds-Karp oraz SAP.

\begin{figure}[H]
    \centering
    \includegraphics[width=0.8\textwidth]{comparison_task1.png}
    \caption{Porównanie czasu wykonania i liczby ścieżek powiększających dla algorytmów EK i SAP w funkcji wymiaru $k$.}
    \label{fig:task1}
\end{figure}

\textbf{Wnioski:}
\begin{itemize}
    \item Wraz ze wzrostem wymiaru $k$, liczba wierzchołków rośnie wykładniczo ($2^k$), co drastycznie zwiększa czas obliczeń.
    \item Algorytm SAP wykazuje na ogół lepszą wydajność dla dużych grafów dzięki lepszemu zarządzaniu odległościami w grafie rezydualnym.
    \item Liczba znalezionych ścieżek powiększających jest skorelowana z wartością maksymalnego przepływu.
\end{itemize}

\subsection{Zadanie 2: Skojarzenia w grafie dwudzielnym}
Zbadano wpływ stopnia wierzchołka $i$ na wielkość maksymalnego skojarzenia w losowym grafie dwudzielnym o $2^k$ wierzchołkach po każdej stronie.

\begin{figure}[H]
    \centering
    \includegraphics[width=0.8\textwidth]{task2_matching_vs_i.png}
    \caption{Rozmiar maksymalnego skojarzenia w zależności od stopnia $i$ dla różnych $k$.}
    \label{fig:task2_matching}
\end{figure}

\begin{figure}[H]
    \centering
    \includegraphics[width=0.8\textwidth]{task2_time_vs_k.png}
    \caption{Czas wykonania algorytmu znajdowania skojarzenia w zależności od $k$.}
    \label{fig:task2_time}
\end{figure}

\textbf{Wnioski:}
\begin{itemize}
    \item Dla małych wartości $i$ skojarzenie jest niepełne.
    \item Wraz ze wzrostem $i$ (liczby krawędzi wychodzących z $V_1$), rozmiar skojarzenia szybko dąży do $2^k$ (skojarzenie doskonałe). Istnieje "próg", powyżej którego z dużym prawdopodobieństwem znajdujemy skojarzenie doskonałe.
\end{itemize}

\subsection{Poprawność wyników (GLPK)}
Dla małych instancji ($k \le 4$) wyniki własnych implementacji zostały zweryfikowane przy użyciu solvera GLPK (wczytującego wygenerowane modele LP).
\begin{itemize}
    \item Wartości funkcji celu (Max Flow / Max Matching) były identyczne dla obu metod.
    \item Potwierdza to poprawność implementacji algorytmów grafowych oraz generatora modeli LP.
\end{itemize}

\section{Podsumowanie}
Zrealizowane zadania pozwoliły na praktyczne zapoznanie się z problematyką maksymalnego przepływu. Eksperymenty potwierdziły teoretyczne założenia dotyczące złożoności obliczeniowej oraz pokazały elastyczność redukcji problemów (skojarzenia) do przepływów. Weryfikacja za pomocą programowania liniowego stanowi silny dowód poprawności zaimplementowanych rozwiązań.

\end{document}
